\documentclass[12pt]{article}
\usepackage{latexsym}
\usepackage{amssymb,amsmath}
\usepackage[pdftex]{graphicx}
\usepackage{listings}
\usepackage{xcolor}
\lstset{language=R,basicstyle=\scriptsize\ttfamily,commentstyle=\ttfamily\color{gray},frame=single,breaklines=true,keepspaces = true,keywordstyle=\color{red},xleftmargin=-10mm}


\topmargin = 0.1in \textwidth=5.7in \textheight=8.6in

\oddsidemargin = 0.2in \evensidemargin = 0.2in


\begin{document}


\begin{center}
MATHEMATICS E-156, SPRING 2014 \\
MATHEMATICAL FOUNDATIONS OF STATISTICAL SOFTWARE

\smallskip

Module \#4 (Hypothesis Testing with Chi Square)
\end{center}

Last modified: February 23, 2014 (fixed dog-bite section problem)

\medskip

\paragraph*{Reading from Chihara and Hesterberg}

\begin{itemize}
\item Sections 3.4.2 through 3.8

\item Appedix B.6 (the Poisson distribution)

\item Appendix B.3 (the multinomial distribution)
\end{itemize}

\paragraph*{Optional Reading from Haigh}

\begin{itemize}
\item Page 53 derives the Poisson distribution as the limit of the binomial distribution.
\end{itemize}


\paragraph*{Proof of the Week}
\begin{itemize}
\item Prove that the Poisson distribution with parameter $\lambda$ has mean and variance both equal to $\lambda$. Prove that if $X_1$ and $X_2$ are independent Poisson random variables with parameters $\lambda_1$ and $\lambda_2$ respectively, then $X_1 + X_2$ is Poisson with parameter $\lambda_1 + \lambda_2$.

(This is theorems B.5 and B.6 on page 380)
\end{itemize}

\pagebreak


\paragraph*{R scripts}
\begin{itemize}
\item 4A-ChiSquare.R\\
Topic 1 -- when should we expect a chi-square distribution?\\
Topic 2 -- why do we sum over all the cells in the table?\\

\item 4B-Homogeneity.R

Topic 1 -- creating a contingency table and doing a chi-square test\\
Topic 2 -- doing a permutation test with a chi-square statistic\\

\item 4C-GoodnessFit.R

Topic 1 - comparing data with a probability model

\item 4D-GoodnessFitDistrib.R

Topic 1 - hypothesized distribution has all parameters specified\\
Topic 2 - hypothesized distribution has a parameter to be estimated from the data

\item 4E-Multinomial.R

Topic 1 - installing and using the multinomial test\\
Topic 2 -- using the multinomial test on examples from the textbook

\item  4P-Proof 4.R

Topic 1 - Poisson distribution as the limit of the binomial distribution \\
Topic 2 -- the sum of two Poisson distributions

\item 4X-Hypergeometric.R (optional)

Topic 1 - some fabricated examples witth small numbers\\
Topic 2 -- some real data: do Bush voters own guns?

\end{itemize}

\pagebreak





\paragraph*{Mathematical notes}



\begin{enumerate}

\item Poisson distribution as the limit of a binomial distribution

Random variable $X_n$ has a binomial distribution with parameters $n$ and $p$, expectation $\lambda = np$. So $p = \lambda / n.$

Its mass function is $$P(X_n = x) = \binom{n}{x}p^x(1-p)^{n-x}=  \frac{n!}{x!(n-x)!} \frac{\lambda^x}{n^x} (1-\frac{\lambda}{n})^{n-x}.$$

Take the limit as $n \rightarrow \infty$ to get the mass function for a Poisson random variable $X$:

$$P(X = x) = e^{-\lambda}\frac{\lambda^x}{x!}.$$


\pagebreak


\item Proof of the week
\begin{enumerate}


\item Prove that the Poisson distribution with parameter $\lambda$ has mean and variance both equal to $\lambda$.

\item Prove that if $X_1$ and $X_2$ are independent Poisson random variables with parameters $\lambda_1$ and $\lambda_2$ respectively, then $X_1 + X_2$ is Poisson with parameter $\lambda_1 + \lambda_2$.

\end{enumerate}

\pagebreak

\item Multinomial distribution

\begin{enumerate}

\item Suppose that we have $n$ objects of $r$ different types. Show that the number of ways of selecting $x_1$ objects of type 1, $x_2$ objects of type 2, $\cdots$ $x_r$ objects of type $r$ is 
$$\frac{n}{x_1!x_2! \cdots x_r!}.$$

\item Prove that if the probability that an object is of type $i$ is $p_i$, then the probabillity that $x_1$ objects are of type 1, $\cdots$ $x_r$ objects are of type $r$ is

$$P(X_1 = x_1, X_2 = x_2, \cdots X_r = x_r )= \frac{n!}{x_1!x_2! \cdots x_r!} p_1^{x_1} p_2^{x_2}\cdots p_r^{x_r}.$$


\end{enumerate}

\end{enumerate}

\pagebreak


\paragraph*{Section problems}

\begin{enumerate}

\item Exercise 15 on page 72:

For the Flight Delays Case Study in Section 1.1, conduct a test of homogeneity to determine if there is a relationship between carrier and the number of flights delayed more than 30 minutes (\verb!Delayed30!). The answer is on page 401.

\item According to Stirzaker, \emph{Elementary Probability}:\\
``During 1979, in Bristol, 1103 postmen sustained 215 dog bites. A total of 191 postmen were bitten, of whom 145 were bitten just once."

A member of the class has pointed out that this is impossible, since the other 46 postmen were bitten at least twice, making the total bites at least 237.


So change the data to

``During 1979, in Bristol, 1103 postmen sustained 315 dog bites. A total of 191 postmen were bitten, of whom 145 were bitten just once."



Let random variable $X$ be the number of dog bites sustained by an individual postman. If a dog, having decided to bite, chooses its victim at random, $X$ will have a binomial distribution that is very well approximated by a Poisson distribution with $\lambda = 315/1103$. Carry out a goodness-of-fit test for this model.

\item The file PopDensity.csv contains the population density (per square mile) for every city in the U.S. whose poplulation is 100000 or more.

\begin{enumerate}
\item For the last digit, count the number of occurrences  of each of the five pairs of digits 0-1, 2-3, 4-5, 6-7, 8-9 to get a table like the one at the top of page 64, then carry out a test of the hypothesis that the last digits are distributed uniformly. Hint: \verb!234%%10! is equal to 4.

\item For the first digit, count the number of occurrences of each of the three triples of digits 1-2-3, 4-5-6, 7-8-9 to get a table like the one at the top of page 64, then carry out a test of the hypothesis that the first digits are distributed uniformly. Hint: \verb!substr(as.character(234),1,1)! is the character ``2".

\end{enumerate}

\end{enumerate}

\pagebreak


\paragraph*{Homework assignment}

This assignment should be submitted as a single R script. Include enough comments so that it is clear what you are doing and where each problem begins. You can upload it to the dropbox on the Class 4 page of the Web site. 

It is OK to paste and edit lines from the scripts on the course Web site. It is not OK to paste lines from your classmates' solutions!

The first two problems have odd numbers. Feel free to check your answers against page 401.

\begin{enumerate}
\item Exercise 25 on page 73 of the textbook. Solve the problem by doing a chi-square goodness-of-fit test, then solve it by assuming all numbers equally likely and drawing lots of random samples of 500 numbers.

\item Exercise 21 on page 72. 

For the given pdf, it is easy to integrate the density function by standard calculus techniques to find the probabilities for each interval, as was done on page 65. Or you can have R do numerical integration.

 However, the given distribution is a special case of the "Pareto distribution," which is supported by R. If you execute these three lines:

\verb!install.packages("actuar") #comment this out after the first time!

\verb!library(actuar)! 

\verb!help("ppareto")!

you will be able to have R compute the probabilities for the different intervals by subtraction.

\item Exercise 14 on page 71 Do this two ways:

\begin{enumerate}
\item Replicate the contingency table (including the row and column names), then use the built-in chi-square test. 
\item Create a data frame with two columns, then do a permutation test by permuting the Sex column, using chi square as a measure of distance between observed and expected values.
\end{enumerate}

\item Exercise 22 on page 73. The fifty numbers are in the file Exercise22.csv.

\item For the Red Sox data in RedSox2013.csv, consider the number of runs per game scored by the Red Sox. Model this using a Poisson distribution, and perform a goodness-of-fit test to compare the model with the empirical data.

\end{enumerate}









\pagebreak




















\end{document}
